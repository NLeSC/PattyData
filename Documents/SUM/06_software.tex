\section{Software}
\label{sec:software}
As described in section \ref{sec:sys_arch} the 4D GIS system has a two tier
architecture, the server and the clients. In this section we identify the software
required for each of the tiers.

\subsection{Server software}
The \textit{viaappiadb} database running in the Via Appia Linux server is a
PostgreSQL 9.2.8 with PostGIS 2.1.2 and GDAL 1.10.0.

Specific point cloud libraries are also used for the management and processing
of point cloud data, concretely LASzip, libLAS and LAStools. In LAStools and
libLAS many of their applications share the same name and in these cases the
usage of the LAStools ones is preferred. In order to guarantee this we need to
set the \textit{PATH} environment variable accordingly, i.e. by setting the
LAStools bin folder prior to the libLAS one.

As described in section \ref{sec:data_structure} the data stored in the server
needs to be converted to the specific formats required by the two supported
visualizations, i.e. the web viewer based on Potree and the Windows desktop
viewer/editor based on Open Scene Graph. To perform these conversions we use
the \textit{PotreeConverter} ({\url{https://github.com/potree/PotreeConverter}}) and
the \textit{OSGConverter}
(https://github.com/NLeSC/PattyData/tree/master/OSG/converter) which requires the
Open Scene Graph (we currently use 3.2.1). Both converters use Boost (our
installed version is 1.55). The data conversion is controlled through python
scripts. Hence, it is required python bindings for GDAL, LibLAS and LasZIP. 

The web-based viewer requests files from a file server. An NGINX web server is
also used to serve files (static content) to the web-based visualization.

\subsection{Client software}
The web visualization viewer only requires the installation of a modern web
browser in the client laptop/desktop (we have used Chrome/Chromium). For the
desktop-based visualization the user should follow the instructions
at {\url{https://github.com/NLeSC/Via-Appia}}.
