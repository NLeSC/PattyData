\newpage
\section{Software}
\label{sec:software}
As described in section \ref{sec:sys_arch} the 4D GIS system has a two tier
architecture, the server and the clients. 

\subsection{Server software}
The \textit{viaappiadb} database running in the Via Appia Linux server is a
PostgreSQL 9.2.8 with PostGIS 2.1.2 and GDAL 1.10.0.

Specific point cloud libraries are also used for the management and processing
of point cloud data, concretely LASzip, libLAS and LAStools. In LAStools and
libLAS many of their applications share the same name and in these cases the
usage of the LAStools ones is preferred. In order to guarantee this we need to
set the \textit{PATH} environment variable accordingly, i.e. by setting the
LAStools bin folder prior to the libLAS one.

As described in section \ref{sec:data_structure} the data stored in the server
needs to be converted to the specific formats required by the two supported
visualizations, i.e. the web viewer based on Potree and the Windows desktop
viewer/editor based on Open Scene Graph. To perform these conversions we use
the \textit{PotreeConverter} (https://github.com/potree/PotreeConverter) and
the \textit{OSGConverter}
(https://github.com/NLeSC/Via-Appia/tree/master/converter) which requires the
Open Scene Graph (we currently use 3.2.1). Both converters use Boost (our
installed version is 1.55).

An NGINX web server is also used to serve files (static content) to the
web-based visualization. 

\subsection{Client software}
The web visualization viewer only requires the installation of a modern web
browser in the client laptop/desktop (we have used Chrome/Chromium).

For the desktop-based visualization we use a tool based on OpenSceneGraph (OSG)
so the point cloud data and also the pictures and meshes have to be converted
to OSG format. This conversion also happens in the Via Appia server and it is
executed by a

For the usage of the OSG desktop viewer some additional libraries have to be
installed

{\em pasted form old documentaiton- to be edited!}
 
In order to use the new 4D GIS system the end-user only needs to do:
\begin{itemize} \item (a.) Obtain an account in the Via Appia Linux server and
in the \textit{viaappiadb} database and generate a SSH key pair (see
Subsections \ref{sec:accounts} and \ref{sec:sshkeys}).  \item (b.) Download,
install and configure the \textit{launcher} tool in his Windows laptop or
desktop (see Subsection \ref{sec:install}).  The \textit{launcher} tool also
contains the 4D viewer. It is also recommend to install two tools for the
communication between the Linux server and the Windows computers: PuTTY and
WinSCP (see Subsubsections \ref{sec:putty} and \ref{sec:winscp}).
\end{itemize}

The database is a PostgreSQL 9.2.8 (version correct?) instance running with
PostGIS 2.1.2.

We also need the python binding of GDAL and LiblAS.
