\section{Future work}
\label{sec:future_work}
The foundations for a 4D GIS for Point Cloud data exploration were successfully
designed and implemented. In this section we summarize the major future steps
to make the system more user efficient, robust and user friendly.

\paragraph{Efficiency}
As the number of different objects grows and the number of users increases, it
is necessary to have multi-thread scripts for data manipulation, conversion and
database operations. For example, the potree converter should be multi-thread.
The same holds for database access. Multi-session should be exploited as well as
query caching. 

On the fly data generation would also open the opportunity to easily integrate
new data without having to re-convert all the raw data for a background or site.
In the same line, different data organization techniques for Potree, instead
the actual octree, should be exploit in case they easy, or boost, on the fly
data generation and visualization.

\paragraph{Robustness}
At the moment concurrent access for object manipulation is not controlled. For
example, in case more than one user changes the object location it leads to data
corruption. Hence, as first step it is required a feature which blocks, or discards,
other users modifications while the object is being modified. The second step is
to allow parallel object manipulation controlled by a version control mechanism.

The addition of new features and the concurrent access by multiple users is a
source of robustness issues. The current functionality must remain intact with
the addition of new features. Currently the a basic test platform was introduced.
However, it is not yet automatized and the set of unit tests is small. In the
future all the features, and bug reports, should be covered by unit tests.

\paragraph{User friendly}
The data management, such as data upload and data transformation, should be done through
a single user interface (UI). The same holds for data visualization. The data
conversion and database updates should be triggered by visualization requests
on the viewers.

The ultimate goal is an end to end solution where all the parts are glued and
automated. Such system should allow an archeologist to be on the field
upload a set of photos to the viappia server and wait for an email notification which
informs the archeologist about the successful integration of the new data into
an existent one. Then with one click the archeologist should be able to 
visualize/manipulate the new data through the 4D viewer.
