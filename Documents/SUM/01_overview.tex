\section{Overview}
\label{sec:dataman_overview}

The subject of this project is an archaeological study of an area full of
funerary monuments between the fifth and sixth miles of the Via Appia
Antica ({\url{http://en.wikipedia.org/wiki/Appian_Way}}).
The data gathered from this area is of different resolutions and types: point clouds, meshes, photos, historical images, 2D footprints and
attribute information of the monuments. The data is managed,
processed and visualized by the Via Appia 4D Geospatial Information Systems (GIS).
Such system combines a Windows desktop viewer, a web-based viewer and a Database Management System (DBMS).

Through a client-server architecture, multiple researchers at different locations
are able to analyze and study different areas of the ViaAppia. Such architecture creates the base for 4D GIS systems for the exploration of large and complex
archaeological areas.

This document is the user manual of the developed Via Appia 4D GIS system. The remainder of it is organized as follows: the data are heterogeneous in nature due to both their conceptual difference and acquisition method. The conceptual data description can be found in Section~\ref{sec:concept_descr}. The client-server architecture of the system is described in Section~\ref{sec:sys_arch}. The storage of the data is
organized according to a predefined hierarchical structure as described in
Section~\ref{sec:data_structure}. The database storing information about the location  of the data as
well as many attribute data (meta-data) of archaeological interest is
explained in Section~\ref{sec:database}. The software used in the 4D GIS is
described in Section~\ref{sec:software}. The report finalizes with an overview
of future steps to be taken in Section~\ref{sec:future_work}
