\section{Overview}
\label{sec:dataman_overview}

 The subject of this project is an archaeological study of an area full of funerary monuments between the fifth and sixth miles of the Via Appia Antica. The data gathered from this area are of different resolution and different types (modalities): point clouds, meshes, photos, historical images and attribute information of the monuments of interest. The data are managed, processed and visualized by the Via Appia 3D GIS. The GIS combines a 3D viewer and a database (DB) and allows the user to refine which data are visualized based on some attributes. The system, which can be updated when new data is available, allows multiple researchers in different locations to analyze and study the area in 3D and aims to be an example for other complex archaeological study areas. This document is the user manual for the data management part of the system.

The Via Appia 3D GIS has client-server architecture described in Section \ref{sec:}.The data are heterogeneous in nature both due to their conceptual difference and acquisition modality. The conceptual data description can be found in Section \ref{sec:concept_descr}. The storage of the data is organized according to a predefined hierarchical convention as described in Section \ref{sec:data_structure}. The DB, storing the actual data location as well as many attribute data (meta-data) of archaeological interest, is explained in Section \ref{sec:}.   