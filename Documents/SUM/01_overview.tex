\section{Overview}
\label{sec:dataman_overview}

The subject of this project is an archaeological study of an area full of
funerary monuments between the fifth and sixth miles of the Via Appia Antica~\cite{}.
The data gathered from this area is of different resolution and different
types (modalities): point clouds, meshes, photos, historical images and
attribute information of the monuments of interest. The data is managed,
processed and visualized by a Via Appia 4D Geospatial Information Systems (GIS).
Such system combines two 4D viewers and a Database Management System (DBMS).

Through a client-server architecture, multiple researchers at different locations
are able to analyze and study different areas of the ViaAppia. Using the data
attributes the user requests a data sub-set from the DBMS and visualizes it.
Such architecture creates the base 4D GIS data exploration on large and complex
archaeological study areas.

This document is the user manual and the remainder of it is organized as follows.
The Via Appia 4D/4D GIS has client-server architecture described in Section
\ref{sec:sys_arch}.The data are heterogeneous in nature both due to their
conceptual difference and acquisition modality. The conceptual data description
can be found in Section \ref{sec:concept_descr}. The storage of the data is
organized according to a predefined hierarchical convention as described in
Section \ref{sec:data_structure}. The database storing the actual data location as
well as many attribute data (meta-data) of archaeological interest is
explained in Section \ref{sec:database}. The software used in the 4D/4D GIS is
described in Section \ref{sec:software}.
